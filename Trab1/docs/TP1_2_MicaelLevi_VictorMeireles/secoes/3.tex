\section{Programas Implementados}\label{sec:programas_implementados}
% Antes de detalhar o objetivo de cada fonte, segue uma tabela com as informações listando as funções que foram efetivamente utilizadas.
% \begin{table}[!htbp]
  \centering
    \begin{tabular}{l|l}
    \rowcolor[rgb]{ .647,  .647,  .647} \multicolumn{1}{c|}{\textcolor[rgb]{ 1,  1,  1}{\textbf{fonte .hpp}}} & \multicolumn{1}{c}{\textcolor[rgb]{ 1,  1,  1}{\textbf{funções}}} \\
    \midrule
    \rowcolor[rgb]{ .859,  .859,  .859} artigo & getArtigoId, clearFieldIfInvalid, getNextFieldFrom, getRecordArtigoFrom \\
    \midrule
    \rowcolor[rgb]{ .929,  .929,  .929} log   & errorMessageExit, basicMessage \\
    \midrule
    \rowcolor[rgb]{ .859,  .859,  .859} stringUtils & stringToInt, strcpy \\
    \end{tabular}%
  \caption{A fonte e suas funções que foram usadas.}
  \label{tab:codigos_funcoes}%
\end{table}%

O objetivo, autor e descrição de cada função implementada estão listados na seção \subsecao{descricao_das_fontes}.
Abaixo explicaremos brevemente o conteúdo dos códigos-fonte.

O local e nome dos arquivos binários que serão gerados pelos programas está definido no código-fonte \texttt{include/parametros.h}, $\ie$ contém informações que serão utilizadas por todos os programas implementados.


O código \texttt{include/artigo.hpp} contém valores específicos para o tipo de registro deste trabalho.
E a estrutura que comportará tal registro está implementada como:
\includecode{codigos/struct_artigo.h}

Os campos foram dispostos em ordem crescente pelo tamanho do campo para reduzir o \emph{structure padding}\footnote{Inserção de ``gaps'' pelo compilador para alinhar os membros da estrutura ao seu endereço natural, que deve ser múltiplo do comprimento (em bytes) do campo.\citar{haase}}. Os tamanhos estão como macros no próprio arquivo para tornar o código mais legível e dinâmico, e serão definidos em tempo de compilação. Por padrão possuem os valores descritos na tabela \tabela{tipo_registro}.

O código \texttt{include/externalHash.hpp} implementa a classe \emph{ExternalHash} que, como o nome indica, manterá a hash externa a ser criada.
Além de uma estrutura para representar um bloco de memória (ou página\footnote{Como é chamado um bloco que está na memória principal.}).
Ela está definida no arquivo \texttt{externalHash.hpp} pois só será utilizada pela hash externa:
\includecode{codigos/struct_bloco.h}

Para compilar todos os programas implementados, basta executar o seguinte comando na linha de comandos:
\shellcmd{make}

\newpage

Ainda podemos setar os valores de algumas macros.
Estas estão dispostas na tabela \tabela{macros_make} abaixo.
\begin{table}[!htbp]
  \centering
    \begin{tabular}{|c|c|c|}
    \toprule
    \rowcolor[rgb]{ .816,  .808,  .808} \textbf{nome} & \textbf{valor padrão} & \textbf{descrição} \\
    \midrule
    DEBUG & false & Modo depuração \\
    \midrule
    TEST & false & Usar a versão reduzida \\
    \midrule
    $QTD\_BUCKETS$ & $1549147$ & Número de buckets alocados \\
    \midrule
    $QTD\_BLOCOS\_POR\_BUCKET$ & $1$ & Número de blocos por bucket \\
    \bottomrule
    \end{tabular}%
  \caption{Variáveis definidas em tempo de compilação ($\eg$ \texttt{make TEST=true}).}
  \label{tab:macros_make}%
\end{table}%


\subsection{Descrição das Fontes}\label{subsec:descricao_das_fontes}

Em termos gerais, seguem as seguintes explicações sobre as fontes desenvolvidas.

No arquivo \texttt{include/artigo.hpp}, temos as funções:
\begin{enumerate} %%%%%%%% especificação do arquivo artigo.hpp
\setlength\itemsep{1em}

\item \texttt{\palavraReservada{int} \nomeFuncao{getArtigoId}(\palavraReservada{const} Artigo\& artigo);}\\
    \noindent \textbf{autor:} Victor Meireles \\
    \noindent \textbf{objetivo:} Retornar o id de um dado artigo (getter id)


\item \texttt{\palavraReservada{static void} \nomeFuncao{clearFieldIfInvalid}(\palavraReservada{char}* field);} \\
    \noindent \textbf{autor:} Micael Levi \\
    \noindent \textbf{objetivo:} Setar o caractere nulo na posição $0$  da string caso o seu conteúdo seja ``inválido''\footnote{Uma string é dita inválida se o seu conteúdo for igual a string ``-1''.}, $\ie$, campo sem valor ou com lixo de memória.


\item \texttt{\palavraReservada{static char}* \nomeFuncao{getNextFieldFrom}(std::istream\& inputStream, \palavraReservada{const char}\& delimiter = \caractere{`;'});} \\
    \noindent \textbf{autor:} Victor Meireles \\
    \noindent \textbf{objetivo:} Ler um campo de um registro do arquivo texto


\item \texttt{Artigo* \nomeFuncao{getRecordArtigoFrom}(std::istream\& inputStream);} \\
    \noindent \textbf{autor:} Victor Meireles \\
    \noindent \textbf{objetivo:} Ler um registro do arquivo texto

\end{enumerate}

\newpage
O arquivo \texttt{include/externalHash.hpp} contém a classe \emph{ExternalHash} que possui os seguintes métodos públicos:
\begin{enumerate} %%%%%%%% especificação do arquivo externalHash.hpp
\setlength\itemsep{1em}

\item \texttt{\palavraReservada{void} \nomeFuncao{create}(\palavraReservada{void});} \\
    \noindent \textbf{autor:} Micael Levi \\
    \noindent \textbf{objetivo:} Criar o arquivo de dados organizado em hash estático (deixa a stream aberta para uso futuro)


\item \texttt{\palavraReservada{void} \nomeFuncao{closeStream}(\palavraReservada{void});} \\
    \noindent \textbf{autor:} Micael Levi \\
    \noindent \textbf{objetivo:} Fechar a stream do arquivo de dados


\item \texttt{\palavraReservada{void} \nomeFuncao{deleteHashfile}(\palavraReservada{void});} \\
    \noindent \textbf{autor:} Micael Levi \\
    \noindent \textbf{objetivo:} Fechar a stream do arquivo de dados e apagá-lo


\item \texttt{\palavraReservada{bool} \nomeFuncao{insertRecordOnHashFile}(\palavraReservada{const} TypeRecord\& record);} \\
    \noindent \textbf{autor:} Micael Levi \\
    \noindent \textbf{objetivo:} Inserir um registro no arquivo de dados criado


\item \texttt{\palavraReservada{unsigned long} \nomeFuncao{findRecord}(TypeRecord\& record, \palavraReservada{bool}\& recordFind);} \\
    \noindent \textbf{autor:} Micael Levi \\
    \noindent \textbf{objetivo:} Procura um registro de acordo com o valor do campo de hashing

\end{enumerate}


As funções que foram utilizadas do \emph{namespace} Log, estão no arquivo \texttt{include/log.hpp}. São estas:
\begin{enumerate} %%%%%%%% especificação do arquivo log.hpp
\setlength\itemsep{1em}

\item \texttt{\palavraReservada{template}$<$\palavraReservada{int} code = EXIT\_FAILURE, \palavraReservada{typename}... Args$>$ \\ \palavraReservada{void} \nomeFuncao{errorMessageExit}(Args... args);} \\
    \noindent \textbf{autor:} Micael Levi \\
    \noindent \textbf{objetivo:} imprimir uma mensagem de erro referente ao código de erro passado no parâmetro
    
\item \texttt{\palavraReservada{template}$<$\palavraReservada{typename}... Args$>$\\ \palavraReservada{void} \nomeFuncao{errorMessageExit}(Args... args);} \\
    \noindent \textbf{autor:} Micael Levi \\
    \noindent \textbf{objetivo:} Imprimir uma mensagem comum de log

\end{enumerate}
\subsection{Programa \texttt{upload}}\label{subsec:programa_upload}

O código-fonte \texttt{src/upload.cpp} contém apenas uma simples função \texttt{main} que utiliza a interface disposta pela classe \emph{ExternalHash} para realizar o processo de criação do arquivo e alocação do espaço de endereçamento da hash externa através do método \texttt{create}. E logo depois faz o uso de uma função implementada em \texttt{include/artigo.hpp} para ler os registros do arquivo.

Após compilar os programas, para executar o \texttt{upload}, temos que passar o nome do arquivo texto que contém os dados. A forma de uso é esta:
\shellcmd{upload <path/to/data.csv>}
\subsection{Programa \texttt{findrec}}\label{subsec:programa_findrec}

O código-fonte \texttt{src/findrec.cpp} contém também apenas uma simples função \texttt{main} que, assim como o programa \texttt{upload}, utiliza a classe \emph{ExternalHash}. Porém, faz o uso apenas do método \texttt{findRecord}. 

Para executar esse programa, devemos passar o número do ID do registro a ser buscado. E, opcionalmente, definir o arquivo de dados que será utilizado.
A forma de uso é esta:
\shellcmd{findrec <record\_id>  [path/to/hashfile]}



%%%%%%%%%%%%%%%%%%%%%%%%%%%%%%%%%%%%%%%%%%%%%%%%%%%%%%%%%%%