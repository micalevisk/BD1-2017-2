\section{Decisões de Projeto}\label{sec:decisoes_de_projeto}

Primeiro, admitimos que a estrutura do diretório está disposta da seguinte maneira:
\\
% (c) https://tex.stackexchange.com/questions/328886/making-a-directory-tree-of-folders-and-files
\begin{forest}
    pic dir tree,
    where level=0{}{% folder icons by default; override using file for file icons
        directory,
    },
    [TP1\underscore 1\underscore MicaelLevi\underscore VictorMeireles
      [docs
        [generatedfiles \comentario{local dos arquivos que serão gerados}
        ]
      ]
      [include
        [artigo.hpp, file
        ]
        [externalHash.hpp, file
        ]
        [log.hpp \comentario{lib}, file
        ]
        [parametros.h \comentario{guarda valores comun aos programas}, file
        ]
        [stringUtils.hpp \comentario{lib}, file
        ]
      ]
      [Makefile \comentario{para compilar os códigos-fonte},file
      ]
      [src
        [program\underscore findrec
          [findrec.cpp, file
          ]
        ]
        [program\underscore upload
          [upload.cpp, file
          ]
        ]
      ]
    ]
\end{forest}


Os arquivos identificados como ``lib'' contém códigos utilitários que foram desenvolvidos para auxiliar em algumas tarefas dos programas implementados. Eles possuem mais informação do que o necessário para os programas finais apenas para fins de reúso em projetos futuros.
O objetivo de cada código-fonte, e demais decisões, foram descritas no cabeçalho dos mesmos, além da documentação \emph{in code} das classes, funções/métodos e variáveis/atributos, seguindo o estilo JavaDoc.
Além disso, os códigos-fonte obedecem as dependências descritas na figura \figura{dependencias}.

Vale ressaltar que o arquivo que contém a definição de um \emph{Artigo} -- tipo do registro do problema proposto -- não possui relação alguma com a classe desenvolvida para manter a hashing externa\footnote{Nome designado para hashing de arquivos em disco.\citar{navathe}} pois esta última fora desenvolvida de maneira neutra (genérica) pensando no reutilização em outros códigos. A quantidade de blocos por bucket e a quantidade de buckets que serão reservados podem ser definidos em tempo de compilação através de algumas diretivas de pré-processamento (macros) que estão disposta na tabela \tabela{macros_make}.
\begin{figure}[h!tbp]
\centering
  \includegraphics[width=1\linewidth]{imagens/dependencias_TB.png}
\caption{Relações de dependências entre os códigos-fonte.}
\label{fig:dependencias}
\end{figure}


\subsection{Estrutura do Arquivo de Dados}\label{subsec:estrutura_do_arquivo_de_dados}

Para o arquivo de dados, optamos por organizá-lo em \textbf{hash estático} pois os registro possuem um tamanho fixo $R = 2384$ bytes. Assim, como um bloco tem tamanho $B = 4096$ bytes e usando a \textbf{alocação não espalhada}\footnote{Os registros não podem atravessar os limites de bloco.\citar{navathe}} como política de alocação dos registros em um bloco, obtemos o seguinte fator de blocagem $bfr$ (visto que $B \geq R$):
\begin{equation}\label{eq:binomial_distribution}
    bfr = \floor{\frac{B}{R}} = 1
\end{equation}

Então teremos até 1 registro por bloco alocado. Tal decisão implicou em um desperdiço de $B - (bfr * R) = 1704$ bytes ($41,8\%$) por bloco mas simplificou a implementação da inserção e da leitura dos registros.

No hashing externo, o espaço de endereços de destino é feito em buckets\footnote{Um bucket é a unidade de armazenamento que pode armazenar um ou mais blocos.\citar{sudarshan}}.
Por se tratar de hash estático, a quantidade $M$ de buckets precisa ser fixada. Definimos um valor de $1549147$ buckets. Que foi escolhido devido a dois fatores:
\begin{itemize}[noitemsep]
    \item Natureza dos registros que serão mapeados
    \item Função de hash
\end{itemize}


Para endereçar um registro ao seu respectivo bucket, utilizamos uma função de hash\footnote{Seja $K$ o conjunto de todos os valores de chave de busca e $Eb$ o conjunto de todos os endereços de bucket, uma função de hash é uma função de $K$ para $Eb$.\citar{sudarshan}} $h$ simples mas que garante uma distribuição uniforme, como mostra a equação \equacao{funcao_hash}, onde $k$ é a chave de busca.
Assim podemos afirmar que não ocorrerão colisões\footnote{Quando um registro é mapeado para um bucket cheio.}, então \textbf{não precisaremos de páginas de overflow}.
\begin{equation}\label{eq:funcao_hash}
    h(k) = k \bmod M
\end{equation}

Sendo $k$ o valor do campo \emph{id} -- pois é uma chave primária -- e $k \in \mathbb{N}$ tal que $1 \leq k \leq 1549146$, a quantidade $M$ de buckets definida anteriormente é suficiente para portar todos os registros sem colisão que serão lidos, já que escolhemos alocar uma quantia $m = 1$ bloco por bucket. Porém, não podem ser inseridos mais do que $1549146$ registros. A distribuição uniforme implica em complexidade $O(1)$ para a busca de registros que estão na hash externa e apenas $1$ acesso a disco.

Com todas essas variáveis definidas, podemos então prever o tamanho do arquivo de dados.
Basta fazer $B * m * M$ que obteremos $6345306112$ bytes, $\ie$ um pouco mais de $6,3$ gigabytes!
E esse é uma outra consequência causada pela simplicidade e eficiência da hash.

Pensando nisso, em nossos testes optamos por utilizar uma \emph{versão reduzida} dos registros. Definimos tamanhos diferentes para os campos descritos na tabela \tabela{tipo_registro} a fim de reduzir para $500$ gigabytes o tamanho do arquivo de dados e ``otimizar'' a criação do mesmo.
Tais alterações estão descritas no código-fonte \texttt{include/externalHash.hpp}.