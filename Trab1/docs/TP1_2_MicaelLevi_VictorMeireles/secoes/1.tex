\section{Introdução}\label{sec:introducao}
% Esta é uma atividade prática cujo objetivo é desenvolver um SGBD. Este relatório expõe os detalhes de implementação bem como as decisões de projeto.

% Esta atividade prática faz parte da disciplina de Bando de Dados I, ministrada pelo Prof. Dr. Altigran Soares, do IComp na Universidade Federal do Amazonas.

% % \include{dadosColetados.tex}

% Segundo \cite{navathe} e \cite{ramakrishnan} e \cite{silberschatz}

% onde:
% \begin{conditions}
%  y     &  altura de lançamento da esfera \\
%  g     &  aceleração da gravidade \\
%  t     &  tempo de duração da queda \\
% \end{conditions}

Esta é uma atividade prática da disciplina de Banco de Dados, ministrada pelo Professor Dr. Altrigran Soares do Instituto de Computação da Universidade Federal do Amazonas (UFAM).

Este trabalho consiste na implementação de programas (em linguagem C++) para armazenamento e pesquisa informações a partir de uma massa de dados a ser fornecida em um arquivo texto formatado.
Os programas desenvolvidos fornecem suporte para a inserção e busca de dados do tipo \emph{Artigo} descrito na tabela \tabela{tipo_registro} abaixo.
% Table generated by Excel2LaTeX
\begin{table}[htbp]
  \centering
    \begin{tabular}{|c|c|c|}
    \toprule
    \rowcolor[rgb]{ .816,  .808,  .808} \textbf{campo} & \textbf{tipo} & \textbf{descrição} \\
    \midrule
    id    & inteiro & Código identificador do artigo \\
    \midrule
    titulo & alfa 300 & Título do artigo \\
    \midrule
    ano   & inteiro & Ano de publicação do artigo \\
    \midrule
    autores & alfa 1024 & Lista dos autores do artigo \\
    \midrule
    citacoes & inteiro & Número de vezes que o artigo foi citado \\
    \midrule
    atualizacao & data e hora & Data e hora da última atualização dos dados \\
    \midrule
    snippet & alfa 1024 & Resumo textual dos dados do artigo \\
    \bottomrule
    \end{tabular}%
  \caption{Tipo dos registros que serão lidos.}
  \label{tab:tipo_registro}%
\end{table}%