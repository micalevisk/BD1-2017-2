\subsection{Descrição das Fontes}\label{subsec:descricao_das_fontes}

Em termos gerais, seguem as seguintes explicações sobre as fontes desenvolvidas.

No arquivo \texttt{include/artigo.hpp}, temos as funções:
\begin{enumerate} %%%%%%%% especificação do arquivo artigo.hpp
\setlength\itemsep{1em}

\item \texttt{\palavraReservada{int} \nomeFuncao{getArtigoId}(\palavraReservada{const} Artigo\& artigo);}\\
    \noindent \textbf{autor:} Victor Meireles \\
    \noindent \textbf{objetivo:} Retornar o id de um dado artigo (getter id)


\item \texttt{\palavraReservada{static void} \nomeFuncao{clearFieldIfInvalid}(\palavraReservada{char}* field);} \\
    \noindent \textbf{autor:} Micael Levi \\
    \noindent \textbf{objetivo:} Setar o caractere nulo na posição $0$  da string caso o seu conteúdo seja ``inválido''\footnote{Uma string é dita inválida se o seu conteúdo for igual a string ``-1''.}, $\ie$, campo sem valor ou com lixo de memória.


\item \texttt{\palavraReservada{static char}* \nomeFuncao{getNextFieldFrom}(std::istream\& inputStream, \palavraReservada{const char}\& delimiter = \caractere{`;'});} \\
    \noindent \textbf{autor:} Victor Meireles \\
    \noindent \textbf{objetivo:} Ler um campo de um registro do arquivo texto


\item \texttt{Artigo* \nomeFuncao{getRecordArtigoFrom}(std::istream\& inputStream);} \\
    \noindent \textbf{autor:} Victor Meireles \\
    \noindent \textbf{objetivo:} Ler um registro do arquivo texto

\end{enumerate}

\newpage
O arquivo \texttt{include/externalHash.hpp} contém a classe \emph{ExternalHash} que possui os seguintes métodos públicos:
\begin{enumerate} %%%%%%%% especificação do arquivo externalHash.hpp
\setlength\itemsep{1em}

\item \texttt{\palavraReservada{void} \nomeFuncao{create}(\palavraReservada{void});} \\
    \noindent \textbf{autor:} Micael Levi \\
    \noindent \textbf{objetivo:} Criar o arquivo de dados organizado em hash estático (deixa a stream aberta para uso futuro)


\item \texttt{\palavraReservada{void} \nomeFuncao{closeStream}(\palavraReservada{void});} \\
    \noindent \textbf{autor:} Micael Levi \\
    \noindent \textbf{objetivo:} Fechar a stream do arquivo de dados


\item \texttt{\palavraReservada{void} \nomeFuncao{deleteHashfile}(\palavraReservada{void});} \\
    \noindent \textbf{autor:} Micael Levi \\
    \noindent \textbf{objetivo:} Fechar a stream do arquivo de dados e apagá-lo


\item \texttt{\palavraReservada{bool} \nomeFuncao{insertRecordOnHashFile}(\palavraReservada{const} TypeRecord\& record);} \\
    \noindent \textbf{autor:} Micael Levi \\
    \noindent \textbf{objetivo:} Inserir um registro no arquivo de dados criado


\item \texttt{\palavraReservada{unsigned long} \nomeFuncao{findRecord}(TypeRecord\& record, \palavraReservada{bool}\& recordFind);} \\
    \noindent \textbf{autor:} Micael Levi \\
    \noindent \textbf{objetivo:} Procura um registro de acordo com o valor do campo de hashing

\end{enumerate}


As funções que foram utilizadas do \emph{namespace} Log, estão no arquivo \texttt{include/log.hpp}. São estas:
\begin{enumerate} %%%%%%%% especificação do arquivo log.hpp
\setlength\itemsep{1em}

\item \texttt{\palavraReservada{template}$<$\palavraReservada{int} code = EXIT\_FAILURE, \palavraReservada{typename}... Args$>$ \\ \palavraReservada{void} \nomeFuncao{errorMessageExit}(Args... args);} \\
    \noindent \textbf{autor:} Micael Levi \\
    \noindent \textbf{objetivo:} imprimir uma mensagem de erro referente ao código de erro passado no parâmetro
    
\item \texttt{\palavraReservada{template}$<$\palavraReservada{typename}... Args$>$\\ \palavraReservada{void} \nomeFuncao{errorMessageExit}(Args... args);} \\
    \noindent \textbf{autor:} Micael Levi \\
    \noindent \textbf{objetivo:} Imprimir uma mensagem comum de log

\end{enumerate}